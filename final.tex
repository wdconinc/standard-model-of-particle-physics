\documentclass[letterpaper,11pt]{article}

\usepackage[utf8x]{inputenc}
\usepackage{enumerate}
\usepackage{enumitem}
\usepackage{fullpage}
\usepackage{amsmath}
\usepackage{siunitx}

\usepackage{pgf}
\usepackage{tikz}
\usetikzlibrary{arrows,shapes,trees}

%opening
\title{Standard Model of Particle Physics \\ Final Exam}

\begin{document}

\maketitle

\paragraph*{Extensions of the Higgs Mechanism}
\begin{enumerate}
  \item \label{higgs:single_multiplet}
  Consider a generalization of the $SU(2) \times U(1)$ Higgs mechanism involving a single multiplet $\phi$ with dimension $2 t + 1$, where $t$ is one of $0, 1/2, 1, 3/2, \ldots$, and whose elements have $T^3$ eigenvalues $t^3 = -t, \ldots, t$. The multiplet has hypercharge $y$ and the electric charge for each component is given by $q = t^3 + y$. Assume the multiplet has an electrically neutral component $\phi^0$ (the component for which $t^3 = -y$). This component acquires a non-trivial vacuum expectation value $\langle \phi^0 \rangle = v/\sqrt{2}$ where $v \ne 0$. Note that the Standard Model is the specific case where $t = 1/2$ and $y = 1/2$.
  \begin{enumerate}
    \item Show that the mass eigenstates $W^\pm$, $Z$, and $A$ in this generalization are the same as in the Standard Model.
    \item Calculate the $W$ and $Z$ masses in terms of $g$, $g'$, $t$, $t^3$, and $v$.
    \item Show that the parameter $\rho_0 = M_W^2 / (M_Z^2 \cos^2\theta_W)$, which is predicted to be unity at tree level in the Standard Model, can be written as
    \begin{equation}
      \rho_0 =  \frac{M_W^2}{M_Z^2 \cos^2\theta_W} = \frac{t(t+1) - (t^3)^2}{2 (t^3)^2}.
    \end{equation}
  \end{enumerate}
  \item \label{higgs:many_multiplets}
  Consider a generalization of the $SU(2) \times U(1)$ Higgs mechanism involving $k$ multiplets $\phi_i, i = 1, \ldots, k$. The dimension of the $i$th multiplet is $2 t_i + 1$, where $t_i = 0, 1/2, 1, 3/2, \ldots$ and the elements have $T^3$ eigenvalues $t^3_i = -t_i, \ldots, t_i$. The $i$th multiplet has hypercharge $y_i$ and the electric charge for each component is given by $q_i = t^3_i + y_i$. Assume each multiplet has an electrically neutral component $\phi^0_i$ which acquires a vacuum expectation value $\langle \phi^0_i \rangle = v_i/\sqrt{2}$ where at least one $v_i \ne 0$.
  \begin{enumerate}
    \item Show that the parameter $\rho_0 = M_W^2 / (M_Z^2 \cos^2\theta_W)$ can now be written as
    \begin{equation}
      \rho_0 =  \frac{M_W^2}{M_Z^2 \cos^2\theta_W} = \frac{\sum_{i=1}^k \left[t_i(t_i+1) - (t^3_i)^2\right] |v_i|^2}{2 \sum_{i=1}^k (t^3_i)^2 |v_i|^2}.
    \end{equation}
  \end{enumerate}
  \item As a specific example of part~\ref{higgs:many_multiplets} consider the extension with two Higgs doublets ($k = 2$, $t_1 = t_2 = 1/2$). This is a feature in minimal supersymmetric extensions of the Standard Model, where in particular
  \begin{equation}
    H_u = \left( \begin{array}{c} h_u^+ \\ h_u^0 \end{array} \right), H_d = \left( \begin{array}{c} h_d^0 \\ h_d^- \end{array} \right).
  \end{equation}
  \begin{enumerate}
    \item Show that this two Higgs doublet model predicts the same value for $\rho_0$ as the Standard Model (at tree level).
    \item Argue that the couplings of neutral physical Higgs bosons to fermions will no longer be flavor-diagonal (in other words, there will be flavor-changing neutral currents).
    \item How many Higgs fields do you expect to survive spontaneous symmetry breaking (similar to the single massive Higgs field in the Standard Model)?
  \end{enumerate}
\end{enumerate}

\paragraph*{Electroweak Sector}
\begin{enumerate}
  \item Consider the decay of a $Z$ boson to pairs of fermions, $Z \to f\bar{f}$. The decay rate to $f_L\bar{f}_L$ differs from that to $f_R\bar{f}_R$ because of the chiral nature of the electroweak interactions. Although chirality eigenstates are not equivalent to helicity eigenstates, for fast-moving fermions they are nearly equivalent. The polarization asymmetry in $Z$ decays to light fermions can be approximated as the asymmetry in decasy to left vs. right-handed fermions.
  \begin{enumerate}
    \item Calculate the left-right asymmetry at tree level for $Z$ decays to a particular fermion $f$, defined by
    \begin{equation}
      A^f_{LR} = \frac{\Gamma(Z \to f_L\bar{f}_L) - \Gamma(Z \to f_R\bar{f}_R)}{\Gamma(Z \to f_L\bar{f}_L) + \Gamma(Z \to f_R\bar{f}_R)},
    \end{equation}
    as a function of the fermion's electric charge and $\sin^2\theta_W$. You may neglect the fermion mass.
    \item Using value from the most recent Particle Data Group's Review of Particle Physics, calculate the left-right asymmetry for charged leptons. Compare for the electron polarization asymmetry quoted in the Review of Particle Physics.
  \end{enumerate}
  \item A top quark decays almost exclusively to a bottom quark and a $W$ boson.
  \begin{enumerate}
    \item Calculate the decay rate $\Gamma(t \to W^+ b)$, assuming $V_{tb} = 1$. Since the $b$ quark is much lighter than the $t$ quark, you may approixmate $m_b = 0$. You should sum over final spins and use the appropriate expression for the sum over $W$ boson spins.
    \item Not assuming $V_{tb} = 1$, calculate the tree level branching ratio for this decay in terms of the CKM matrix elements,
    \begin{equation}
      BR(t \to W^+ b) = \frac{\Gamma(t \to W^+ b)}{\Gamma(t \to W^+ q)},
    \end{equation}
    where in the denominator you should sum over the decay widths to all possible quarks. If possible, simplify using unitarity of the CKM matrix. Estimate the branching ratio using the CKM matrix elements in the most recent Particle Data Group's Review of Particle Physics.
  \end{enumerate}
\end{enumerate}

\paragraph*{The MSW Effect for Neutrinos}
\begin{enumerate}
  \item The propagation of $\nu_e$ and $\nu_\mu$ through matter is affected by interactions with the medium (taken here to be of uniform density). At low energies inelastic scattering can be neglected, so only forward elastic scattering is relevant. An overall phase shift does not matter for flavor changes, so only the difference $\Delta V = V_e - V_\mu$ between the potentials experienced by $\nu_e$ and $\nu_\mu$ is relevant.
  \begin{enumerate}
    \item Demonstrate that in ordinary matter, $\Delta V$ arises from the charged current contribution to $\nu_e e$ elastic scattering, so that $\Delta V = G_F \sqrt{2} n_e$, where $n_e$ is the number density of electrons in the medium.
    \item Show that the Hamiltonian governing the evolution of the $(\nu_e, \nu_\mu)$ system may be written as\
    \begin{equation}
      \left( \begin{array}{cc} -\frac{\Delta m^2}{4E} \cos 2\theta + G_F\sqrt{2} n_e & \frac{\Delta m^2}{4E} \sin 2\theta \\ \frac{\Delta m^2}{4E} \sin 2\theta & \frac{\Delta m^2}{4E} \cos 2\theta \end{array} \right),
    \end{equation}
    where $\Delta m^2 = m_2^2 - m_1^2$ and $\theta$ is the $(\nu_e, \nu_\mu)$ mixing angle in the vacuum.
    \item Compute the mixing parameter $\sin^2 2\theta_m$ in matter, and show that
    \begin{equation}
      \sin^2 2\theta_m = \frac{\sin^2 2\theta}{(\omega - \cos 2\theta)^2 + \sin^2 2\theta},
    \end{equation}
    with $\omega = 2 \sqrt{2} G_F n_e E/\Delta m^2$. Note that $\sin^2 2\theta$ is sensitive to the sign of $\Delta m^2$.
    \item State the conditions for resonant oscillation.
  \end{enumerate}
\end{enumerate}

\end{document}
